%-------------------------
% Resume in Latex
% Author : Shubhi Rani
% License : MIT
%------------------------

\documentclass[letterpaper,10.8pt]{article}

\usepackage{latexsym}
\usepackage[empty]{fullpage}
\usepackage{titlesec}
\usepackage{marvosym}
\usepackage[usenames,dvipsnames]{color}
\usepackage{verbatim}
\usepackage{enumitem}
\usepackage[pdftex]{hyperref}
\usepackage{fancyhdr}


\pagestyle{fancy}
\fancyhf{} % clear all header and footer fields
\fancyfoot{}
\renewcommand{\headrulewidth}{0pt}
\renewcommand{\footrulewidth}{0pt}

% Adjust margins
\addtolength{\oddsidemargin}{-0.375in}
\addtolength{\evensidemargin}{-0.375in}
\addtolength{\textwidth}{1in}
\addtolength{\topmargin}{-.5in}
\addtolength{\textheight}{1in}

\urlstyle{rm}

\raggedbottom
\raggedright
\setlength{\tabcolsep}{0in}

% Sections formatting
\titleformat{\section}{
  \vspace{-3pt}\scshape\raggedright\large
}{}{0em}{}[\color{black}\titlerule \vspace{-5pt}]

%-------------------------
% Custom commands
\newcommand{\resumeItem}[2]{
  \item\small{
    \textbf{#1}{: #2 \vspace{-2pt}}
  }
}

\newcommand{\resumeItemWithoutTitle}[1]{
  \item\small{
    {\vspace{-2pt}}
  }
}

\newcommand{\resumeSubheading}[4]{
  \vspace{-1pt}\item
    \begin{tabular*}{0.97\textwidth}{l@{\extracolsep{\fill}}r}
      \textbf{#1} & #2 \\
      \textit{\small#3} & \textit{\small #4} \\
    \end{tabular*}\vspace{-5pt}
}


\newcommand{\resumeSubItem}[2]{\resumeItem{#1}{#2}\vspace{-4pt}}

\renewcommand{\labelitemii}{$\circ$}

\newcommand{\resumeSubHeadingListStart}{\begin{itemize}[leftmargin=*]}
\newcommand{\resumeSubHeadingListEnd}{\end{itemize}}
\newcommand{\resumeItemListStart}{\begin{itemize}}
\newcommand{\resumeItemListEnd}{\end{itemize}\vspace{-5pt}}

%-------------------------------------------
%%%%%%  CV STARTS HERE  %%%%%%%%%%%%%%%%%%%%%%%%%%%%


\begin{document}

%----------HEADING-----------------
\begin{tabular*}{\textwidth}{l@{\extracolsep{\fill}}r}
  \textbf{{\LARGE Xianzhang Chen}} \\
  Eastern Institute for Advanced Study (EIAS), EITech, Ningbo\\
  No. 568 Tongxin Road, Zhenhai District, Ningbo, Zhejiang, China\\
  % \href{https://www.linkedin.com/in/shubhir/}{Linkedin: https://www.linkedin.com/in/shubhir/} \\ 
  Mobile: +86 18189548536 \\
  Email: xzchen@eitech.edu.cn; chenxzhlzu@outlook.com\\
  % \href{https://github.com/shubhi28}{Github: https://github.com/shubhi28} \\
\end{tabular*}

%-----------EDUCATION-----------------
\section{Education and Empolyment} 
  \resumeSubHeadingListStart
   \resumeSubheading
      {EIAS, EITech}{Ningbo, China}
      {PostDoc in Condensed Matter Theory}{September 2023 -  }
      
      {\textit{Project title: Topological Quantum Computing in Planar Josephson Junction\\
    Advisor: Prof. Tong Zhou}}
      
    \resumeSubheading
      {ICPCS, Lanzhou University}{Lanzhou, China}
      {Ph.D. in Condensed Matter Theory}{September 2016 - June 2023}
      
	{\textit{Thesis title: Wavefunction-related Quantum Transport and Many-body effects in Graphene\\
    Advisor: Prof. Liang Huang and Prof. Rodolfo A. Jalabert}}
      
    \resumeSubheading
      {IPCMS, University of Strasbourg}{Strasbourg, France}
      {Joint training Ph.D. student}{December 2020 - November 2022}
      
      {\textit{Project title: Theory of Scanning Gate Microscopy in Graphene\\
    Advisor: Prof. Rodolfo A. Jalabert}}
	    
    \resumeSubheading
      {School of Physical Science and Technology, Lanzhou University}{Lanzhou, China}
      {Bachelor of Theoretical Physics}{September 2012 - July 2016}

      {\textit{Thesis title: Quantum Transport in an Octagonal Graphene dot\\
    Advisor: Prof. Liang Huang}}
  \resumeSubHeadingListEnd

%
% %--------PROGRAMMING SKILLS------------
% \section{Skills Summary}
% 	\resumeSubHeadingListStart
% 	\resumeSubItem{Languages}{Java, C++, Python, C, SQL, Unix scripting}
% 	\resumeSubItem{Tools}{Kubernetes, Docker, Springboot, GIT, JIRA, Matlab, XCode, Postgres }
% \resumeSubHeadingListEnd

%-----------Awards-----------------
\section{Honors and Awards}
\vspace{0.6em}
\begin{description}[font=$\bullet$]
\item {Selected as a member of Programme Doctoral International (PDI) of Collège doctoral Européen (CDE),  2021}
\item {China Scholarship Council award (EUR 32,400), China Scholarship Council, 2020}
\item {Excellent Poster Award, The 5th National Statistical Physics Conference, 2019}
\item {National Scholarship for Postgraduates (CNY 20,000), Ministry of Education of P. R. China, 2016} 
\end{description}

\section{Personal Statement}
I got my Ph.D. at ICPCS of Lanzhou University with an interest in contributing to a deeper understanding of the classical correspondence of wavefunctions in graphene. The wavefunction is one of the most crucial fundamental concepts in quantum mechanics, which rich physical implications have been extensively debated in the areas of quantum chaos and quantum transport. Graphene has been a focus of experimental and theoretical physics for almost two decades, due to its high mobility at room temperature and the fact that its low-energy quasiparticles can be described by the massless Dirac equation. Consequently, it is important to detect and study the wavefunction properties in graphene. I prefer to do theoretical and numerical analysis that is closely related to experiments and primarily examine scanning gate microscopy (SGM) technology, magnetic transport, and many-body spectral statistics related to the wavefunctions in graphene. After that, I jump to the field topological quantum computing (TQC) with an concentration in the platform of planar Josephson junction (PJJ).

\vspace{0.6em}
\textbf{Research Interests}:
\begin{description}[font=$\bullet$]
\item {Topological Quantum Computing}
\item {Mesoscopic System} 
\item {Quantum Transport Theory}
\item {Scanning Gate Microscopy Technology}
\item {Many-body Physics}
\end{description}

%-----------EXPERIENCE-----------------
\section{Publications}
  % \resumeSubHeadingListStart
  %   \resumeSubheading
  %   {Scanning Gate Microscopy}{}
  %   {Contributions: }{}
  \begin{itemize}[font=$\bullet$]
  \item[]{\textbf{Quantum Transport in Graphene}
    \begin{itemize}
        \item {\textbf{Xian-Zhang Chen}, Guillaume Weick, Dietmar Weinmann, Rodolfo A. Jalabert, \textit{Scanning gate microscopy in graphene nanostructures}, Phys. Rev. B 107, 085420 (2023).\\
        \textit{Contributions: Preliminary analytical derivation, calculation and analysis of numerical results, figure making, and paper writing discussion.}
        }
        \item {Guan-Qun Zhang, \textbf{Xian-Zhang Chen}, Li Lin, Hai-Lin Peng, Zhong-Fan Liu, Liang Huang, Ning Kang, and Hong-Qi Xu, \textit{Transport signatures of relativistic quantum scars in a graphene cavity}, Phys. Rev. B 101, 085404 (2020).\\
        \textit{Contributions: Numerical simulation and analysis of experimental results and supplementary materials writing.}
        }
    \end{itemize}}

    \item[]{\textbf{Many-body Effects in Graphene}
    \begin{itemize}
        \item {\textbf{Xian-Zhang Chen}, Zhen-Qi Chen, Liang Huang, Celso Grebogi, and Ying-Cheng Lai
, \textit{Many-body spectral statistics of relativistic quantum billiard systems}, Phys. Rev. Research 5, 013050 (2023).\\
        \textit{Contributions: Preliminary paper writing, calculation and analysis of numerical results, and figure making. }
        }
    \end{itemize}}

    \item[]{\textbf{Chaotic Scattering (spin transport, entanglement)}
    \begin{itemize}
        \item {Chen-Rong Liu, \textbf{Xian-Zhang Chen}, Hong-Ya Xu, Liang Huang, and Ying-Cheng Lai, \textit{Effect of chaos on two-dimensional spin transport},  Phys. Rev. B 98, 115305 (2018).\\
        \textit{Contributions: Help in writing calculation programs and numerical results discussion.}
        }
        \item {Chen-Rong Liu, Pei Yu, \textbf{Xian-Zhang Chen}, Hong-Ya Xu, Liang Huang, and Ying-Cheng Lai, \textit{Enhancing von Neumann entropy by chaos in spin–orbit entanglement}, Chin. Phys. B 28, 10 (2019).\\
        \textit{Contributions: Numerical results discussion and paper writing checking.}
        }
    \end{itemize}}
    
    \item[]{\textbf{Classical-quantum Correspondence}
    \begin{itemize}
        \item {Xiao-Liang Li, \textbf{Xian-Zhang Chen}, Chen-Rong Liu, and Liang Huang, \textit{Quantization condition of scarring states in complex soft-wall quantum billiards},  Acta Phys. Sin., 69, 080506 (2020).\\
        \textit{Contributions: Help in writing calculation programs, numerical results analysis, and part of figures making.}
        }
    \end{itemize}}
    
\end{itemize}
    % \resumeItemListStart
    %     \resumeItem{}
    %       {Network Fabric Controller is a logically centralized software controller to manage a distributed physical network fabric or a physical network underlay. Designed and developed a library which can be used by any services within Network Fabric Controller to generate events and raise alerts for NFC managed objects. The events and alerts are displayed on the NFC dashboard.}
    %       \resumeItem{Upgrade NFC}
    %       {Designed and developed an over-the-air and air-gapped upgrade mechanism that is used to upgrade the single node Network Fabric Controller cluster.}
    %       \resumeItem{Health Monitoring System}{Designed and developed a monitoring service which is responsible for monitoring the health of all the micro services running inside NFC cluster.}
    %       \resumeItem{CLI framework}{Developed an internal command line interface tool which provides a set of commands specific to Network Fabric Controller projects to get the system health, logs and current resource utilization. It can be easily extended to perform various other actions.}
    %       \resumeItem{Bootstrap NFC}{Network Fabric Controller is composed of several micro services deployed on the Kubernetes pods on a single-node cluster. Designed and implemented the bootstrapping mechanism to package all the services and deploy on the Kubernetes environment.}
    %       \resumeItem{Install/Upgrade/Uninstall NSX agent}{Worked on install, upgrade and uninstall mechanism of NSX agent on workload VMs deployed on NSX cross cloud environment.}
    %       \resumeItem{AppDiscovery}{Worked on application profiling feature which provides visualization and details of which processes inside a workload VM are communicating on the network.}
    %   \resumeItemListEnd
      
  %   \resumeSubheading
		% {Stony Brook University}{Stony Brook, NY}
		% {Research Assistant - Prof. Erez Zadok }{May 2016 -  August 2016}
		% \resumeItemListStart
  %       \resumeItem{System Call Trace Record/Replay}
  %         {Worked on building a trace replayer at system call level to reproduce system call operations that were captured during a specific workload using C, C++, DataSeries. Developed a wrapper class that makes C++ functions callable by strace C code.}
		% \resumeItemListEnd

%     \resumeSubheading
%     {Samsung Research Institute}{Noida, India}
%     {Software Developer Engineer}{Jun 2012 - July 2015}
%     \resumeItemListStart
%     \resumeItem{Android File System}{}
%     \begin{description}[font=$\bullet$]
%     \item {Involved in board bring-up activities for Android Smart phones based on Exynos and Broadcom chipsets on Android version 4.3 Jelly Bean to Android 5.0 Lollipop.}
%     \item {Experienced in porting of File System (FAT, EXFAT, SDCARDFS, EXT4) on Samsung mobile’s proprietary platform.}
%     \item {Enhanced performance of smart phones having low RAM by analyzing performance using blktrace and tuning kernel parameters. The code was merged in around 15 smart phones.}
%     \end{description}
%     \resumeItemListEnd
% \resumeSubHeadingListEnd

%-----------PROJECTS-----------------
\section{Presentations}
  \begin{itemize}[font=$\bullet$]
  \item[]{\textbf{Talks}
    \begin{itemize}
        \item {Many-body spectral statistics of relativistic quantum billiards systems\\
                DPG Meeting 2022, Regensburg, Germany, 09/2022.}
    \end{itemize}}

    \item[]{\textbf{Posters}
    \begin{itemize}
        \item {Transport signatures of scars in graphene dots at low magnetic field\\
                GDR 2426, Aussois, France, 12/2021.}
        \item {Many-body spectral statistics of relativistic quantum billiard systems\\
                The 5th National Statistical Physics Conference (SPCSC), Hefei, China, 07/2019.}
        \item {Transport signatures of relativistic quantum scars in a graphene cavity\\
                International Workshop on Experimental and Theoretical Developments on Complex Quantum Systems, Lanzhou, China, 07/2018.}
    \end{itemize}}
    
\end{itemize}

\section{Participated Grants}
\begin{description}[font=$\bullet$]
\item {Spectral statistics of a class of pseudo-integrable quantum billiards, CNY 630,000, No. 12175090, NSFC, 2021.}
\item {Study of spin anomalous effects in Dirac billiards, CNY 580,000, No. 11775101, NSFC, 2017.}

\end{description}

%-------------------------------------------
\end{document}