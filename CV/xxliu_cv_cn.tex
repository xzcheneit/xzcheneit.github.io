%%%%%%%%%%%%%%%%%%%%%%%%%%%%%%%%%%%%%%%%%
% "ModernCV" CV and Cover Letter
% LaTeX Template
% Version 1.3 (29/10/16)
%
% This template has been downloaded from:
% http://www.LaTeXTemplates.com
%
% Original author:
% Xavier Danaux (xdanaux@gmail.com) with modifications by:
% Vel (vel@latextemplates.com)
%
% License:
% CC BY-NC-SA 3.0 (http://creativecommons.org/licenses/by-nc-sa/3.0/)
%
% Important note:
% This template requires the moderncv.cls and .sty files to be in the same 
% directory as this .tex file. These files provide the resume style and themes 
% used for structuring the document.
%
%%%%%%%%%%%%%%%%%%%%%%%%%%%%%%%%%%%%%%%%%

%----------------------------------------------------------------------------------------
%	PACKAGES AND OTHER DOCUMENT CONFIGURATIONS
%----------------------------------------------------------------------------------------

\documentclass[11pt,a4paper,sans]{moderncv} % Font sizes: 10, 11, or 12; paper sizes: a4paper, letterpaper, a5paper, legalpaper, executivepaper or landscape; font families: sans or roman

\moderncvstyle{casual} % CV theme - options include: 'casual' (default), 'classic', 'oldstyle' and 'banking'
\moderncvcolor{blue} % CV color - options include: 'blue' (default), 'orange', 'green', 'red', 'purple', 'grey' and 'black'

\usepackage{lipsum} % Used for inserting dummy 'Lorem ipsum' text into the template
\usepackage{etaremune}
\usepackage[scale=0.75]{geometry} % Reduce document margins
%\setlength{\hintscolumnwidth}{3cm} % Uncomment to change the width of the dates column
%\setlength{\makecvtitlenamewidth}{10cm} % For the 'classic' style, uncomment to adjust the width of the space allocated to your name
\usepackage{xeCJK}
\setCJKmainfont{STXihei}

\definecolor{cvblue}{RGB}{0,117,210}
%----------------------------------------------------------------------------------------
%	NAME AND CONTACT INFORMATION SECTION
%----------------------------------------------------------------------------------------

\firstname{Xiaoxiong} % Your first name
\familyname{Liu} % Your last name

% All information in this block is optional; comment out any lines you don't need
\title{Curriculum Vitae}
\photo[70pt][0.4pt]{../img/profile-img.jpg} % The first bracket is the picture height; the second is the thickness of the frame around 

%----------------------------------------------------------------------------------------

\begin{document}


%----------------------------------------------------------------------------------------
%	CURRICULUM VITAE
%----------------------------------------------------------------------------------------

\makecvtitle % Print the CV title

%----------------------------------------------------------------------------------------
%	EDUCATION SECTION
%----------------------------------------------------------------------------------------
\section{基本情况}
\cvitem{生日}{02.02.1993}
\cvitem{国籍}{中国}
\cvitem{E-mail}{xxliu@physik.uzh.ch}
\cvitem{职位}{博士研究生,凝聚态理论,苏黎世大学,瑞士}

\section{教育情况}

\cventry{2019--至今}{博士研究生}{苏黎世大学}{瑞士}{}{导师: Dr. Stepan S. Tsirkin}  % Arguments not required can be left empty

\cventry{2016--2019}{硕士 理论物理}{兰州大学}{中国}{}{毕业论文:$F\bar{4}3m$空间群拓扑Weyl半金属的电子性质研究  导师: 邓剑波教授}  

\cventry{2012--2016}{学士 物理}{兰州大学}{中国}{}{毕业论文: 3d过渡金属链参杂岩盐矿结构MgO的第一性原理研究. 导师: 邓剑波教授}  

\section{教学助理}
\subsection{苏黎世大学}
\cvitem{}{负责作业,习题课,答疑}
\cvitem{2022}{量子力学}
\cvitem{2022}{科学研究中的机器学习}
\cvitem{2021}{数学物理方法 I}
\cvitem{2021}{线性代数 II}
\cvitem{2020}{线性代数 I}
\cvitem{2020}{科学计算}
\subsection{兰州大学}
\cvitem{}{负责作业}
\cvitem{2016}{力学}

\section{获奖情况}

\cvitem{2018}{研究生国家奖学金}

\section{语言}

\cvitemwithcomment{中文}{母语}{}
\cvitemwithcomment{英语}{工作语言,流利}{}

\section{相关网页}
\cvitem{个人主页}{\href{https://liu-xiaoxiong.github.io/index.html}{\textcolor{gray}{https://liu-xiaoxiong.github.io/index.html} } }
\cvitem{研究组主页}{\href{https://www.physik.uzh.ch/en/groups/neupert/team/Xiaoxiong-Liu.html}{\textcolor{gray}{https://www.physik.uzh.ch/en/groups/neupert/team/Xiaoxiong-Liu.html} } }
\cvitem{Google学术}{\href{https://scholar.google.com/citations?user=s2Py778AAAAJ&hl=zh-CN&oi=ao}{\textcolor{gray}{https://scholar.google.com/citations?user=s2Py778AAAAJ\&hl=zh-CN\&oi=ao} } }
\cvitem{ResearchGate}{\href{https://www.researchgate.net/profile/Xiaoxiong-Liul}{\textcolor{gray}{https://www.researchgate.net/profile/Xiaoxiong-Liu} } }
\cvitem{Github}{\href{https://github.com/Liu-Xiaoxiong}{\textcolor{gray}{https://github.com/Liu-Xiaoxiong} } }
\cvitem{Gitlab}{\href{https://gitlab.com/Xiaoxiong\_Liu}{\textcolor{gray}{https://gitlab.com/Xiaoxiong\_Liu} } }


\section{科学软件开发情况 (开源)}
\subsection{作者:}
\cvitem{symmetrize wann matrix}{用于对称化Wannier90输出矩阵。 例如,哈密顿矩阵,位置矩阵等。}
\cvitem{}{软件地址: \href{https://github.com/Liu-Xiaoxiong/symmetrize_wann_matrix}{\textcolor{gray}{https://github.com/Liu-Xiaoxiong/symmetrize\_wann\_matrix}} }

\subsection{主要开发者:}
\cvitem{WannierBerri}{一个用于对Berry曲率和磁矩及其偏导进行Wannier差值高级工具。用于输运性质研究。 \href{http://wannier-berri.org}{\textcolor{gray}{http://wannier-berri.org}}}
\cvitem{}{软件地址: \href{https://github.com/wannier-berri/wannier-berri}{\textcolor{gray}{https://github.com/wannier-berri/wannier-berri}} }

\subsection{参与贡献: (正在进行)}
\cvitem{ASE}{是一些列python模块化工具,用于原子结构模拟的初始化,运行,操作,可视和分析。我在负责优化Wannier方程部分。 \href{https://wiki.fysik.dtu.dk/ase/}{\textcolor{gray}{https://wiki.fysik.dtu.dk/ase/}}}

\section{计算机技能}
\cvitemwithcomment{编程语言}{}{Python3, Fortran, Mathematica, Linux}
\cvitemwithcomment{DFT软件}{}{VASP, QuantumEspresso,FPLO,Abnit,Siesta,ASE}
\cvitemwithcomment{DFT后处理}{}{Wannier90,WannierBerri,WannierTools,Irrep,Z2Pack}
\cvitemwithcomment{高通量}{}{AiiDA}

\section{文章发表情况}
\cvitem{}{共发表15篇文章,包括: Nature Material一篇, PRL一篇(共同一作), PRB两篇, APL一篇}
\cvitem{}{引用279次,h-index 7}%----------------------------------------------------------------------------------------
%	AWARDS SECTION
%----------------------------------------------------------------------------------------

\newpage





\section{References}
\cventry{Group Leader}{Titus Neupert}{}{Institut-Physik, University of Zurich, <neupert@physik.uzh.ch>}{}{}
\cventry{supervisor}{Stepan S Tsirkin}{}{Institut-Physik, University of Zurich,  <stepan@physik.uzh.ch>}{}{}
\cventry{co-author}{Ivo Souza}{}{CMF, University of the Basque Country, <ivo\_souza@ehu.eus> >}{}{}


\section{参与会议}
\subsection{口头报告}
\begin{etaremune}
  \item Symmetrization of berry curvature and magnetic moment, \textbf{Wannier 2022 Developers Meeting (smr 3757)}, ICTP, Trieste, Italy, May 23-27, 2022
  \item Gauge-covariant derivatives of the Berry curvature and orbital moment by Wannier interpolation, \textbf{APS March meeting}, Virtual, USA, March 15-19, 2021
\end{etaremune}

\subsection{海报}
\begin{etaremune}
  \item Ab initio calculations of electrical magnetochiral anisotropy with Wannier interpolation, \textbf{Swiss Workshop on Materials with Novel Electronic Properties Basic research and applications}, Les Diablerets, Switzerland, August 29-31, 2022
  \item Ab initio calculations of electrical magnetochiral anisotropy with Wannier interpolation, \textbf{Psi-K Conference}, EPFL, Lausanne, Switzerland, August 22-25, 2022
  \item Systematic study of magnetotransport responses with Berry-Boltzmann formalism, \textbf{First-Principles Modelling of Defects in Solids Workshop}, ETHz, Zurich, Switzerland, June 13-15, 2022
  \item Systematic study of magnetotransport responses with Berry-Boltzmann formalism, \textbf{Wannier 2022 Summer School}, ICTP, Trieste, Italy May 16-20, 2022
  \item Wannier Interpolation of Berry-Boltzmann Formalism for Berry Curvature related quantities with WannierBerri, \textbf{Condensed Matter Theory Symposium}, ETHz, Zurich, Switzerland, September 22, 2021
  \item Gauge-covariant derivatives of the Berry curvature and orbital moment by Wannier interpolation, \textbf{Virtual DPG Spring Meeting}, Virtual, Germany, March 1-4, 2021
  \item Gauge-covariant derivatives of the Berry curvature and orbital moment by Wannier interpolation, \textbf{20th International Workshop on Computational Physics and Materials Science: Total Energy and Force Methods}, Virtual, Italy, February 23-25, 2021
  \item Gauge-covariant derivatives of the Berry curvature and orbital moment by Wannier interpolation, \textbf{Virtual Electronic Structure Workshop}, Virtual, USA, June 3-5, 2020
\end{etaremune}


\section{文章发表}

\begin{etaremune}
  \item Ab initio calculations of electrical magnetochiral anisotropy with Wannier interpolation, \textcolor{cvblue}{Xiaoxiong Liu}, S. S. Tsirkin, I. Souza, in progress.
  \item Systematic study of magnetotransport responses with Berry-Boltzmann formalism, \textcolor{cvblue}{Xiaoxiong Liu}, S. S. Tsirkin, I. Souza, in progress.
  \item Covariant derivatives of Berry-type quantities: Application to nonlinear transport,\textcolor{cvblue}{Xiaoxiong Liu}, M. Á. Jiménez, S. S. Tsirkin, I. Souza, in progress.
  \item Two-dimensional sliding charge density waves and their protected edge modes, SB Zhang, MS Hossain, JX Yin, \textcolor{cvblue}{Xiaoxiong Liu}, MZ Hasan, T Neupert, arXiv preprint arXiv:2204.06269
  \item Origin of spin reorientation and intrinsic anomalous Hall effect in the kagome ferrimagnet TbMn6Sn6, DC Jones, S Das, H Bhandari, \textcolor{cvblue}{Xiaoxiong Liu}, P Siegfried, MP Ghimire, SS Tsirkin, II Mazin, NJ Ghimire, arXiv e-prints, arXiv: 2203.17246
  \item Triple nodal points characterized by their nodal-line structure in all magnetic space groups, PM Lenggenhager, \textcolor{cvblue}{Xiaoxiong Liu}, T Neupert, T Bzdušek, arXiv preprint arXiv:2201.08404
  \item Signatures of Weyl fermion annihilation in a correlated kagome magnet, I. Belopolski, T. A. Cochran, \textcolor{cvblue}{Xiaoxiong Liu}, Z. Cheng, X. Yang, Z. Guguchia, S. S. Tsirkin, J. Yin, P. Vir, G. S. Thakur, S. Zhang, J. Zhang, K. Kaznatcheev, G. Cheng, G. Chang, D. Multer, N. Shumiya, M. Litskevich, E. Vescovo, T. K. Kim, C. Cacho, N. Yao, C. Felser, T. Neupert, M. Z. Hasan, \textbf{Physical review letters} 127 (25), 256403, (2021)
  \item Unconventional chiral charge order in kagome superconductor KV3Sb5, Y. Jiang, J. Yin, M. M. Denner, N. Shumiya, B. R. Ortiz, G. Xu, Z. Guguchia, J. He, M. S. Hossain, \textcolor{cvblue}{Xiaoxiong Liu}, J. Ruff, L. Kautzsch, S. Zhang, G. Chang, I. Belopolski, Q. Zhang, T. A. Cochran, D. Multer, M. Litskevich, Z. Cheng, X. Yang, Z. Wang, R. Thomale, T. Neupert, S. D. Wilson, M. Z. Hasan, \textbf{Nature Materials} 20 (10), 1353-1357, (2021)
  \item Universal higher-order bulk-boundary correspondence of triple nodal points, PM Lenggenhager, \textcolor{cvblue}{Xiaoxiong Liu}, T Neupert, T Bzdušek, arXiv preprint arXiv:2104.11254
  \item From triple-point materials to multiband nodal links, PM Lenggenhager, \textcolor{cvblue}{Xiaoxiong Liu}, SS Tsirkin, T Neupert, T Bzdušek, \textbf{Physical Review B} 103 (12), L121101, (2021)
  \item Intriguing magnetism of the topological kagome magnet $TbMn_6Sn_6$, C Mielke III, Wenlong Ma, V Pomjakushin, O Zaharko, \textcolor{cvblue}{Xiaoxiong Liu}, J-X Yin, SS Tsirkin, TA Cochran, M Medarde, V Poree, D Das, CN Wang, J Chang, T Neupert, A Amato, S Jia, MZ Hasan, H Luetkens, Z Guguchia, arXiv preprint arXiv:2101.05763
  \item Magneto-transport and Shubnikov–de Haas oscillations in the type–II Weyl semimetal candidate NbIrTe4 flake, X. Huang, \textcolor{cvblue}{Xiaoxiong Liu}, P. Yu, P. Li, J. Cui, J. Yi, J. Deng, J. Fan, Z. Ji, F. Qu, X. Jing, C. Yang, L Lu, Z. Liu, G. Liu, \textbf{Chinese Physics Letters} 36 (7), 077101, (2019)
  \item Quantum anomalous Hall effect and topological phase transition in two-dimensional antiferromagnetic Chern insulator NiOsCl6, WW Yang, L Li, JS Zhao, \textcolor{cvblue}{Xiaoxiong Liu}, JB Deng, XM Tao, XR Hu, \textbf{Journal of Physics: Condensed Matter} 30 (18), 185501, (2018)
  \item A nonmagnetic topological Weyl semimetal in quaternary Heusler compound CrAlTiV, \textcolor{cvblue}{Xiaoxiong Liu}, L Li, Y Cui, J Deng, X Tao, \textbf{Applied Physics Letters} 111 (12), 122104, (2017)
  \item Ternary Weyl semimetal $NbIrTe_4$ proposed from first-principles calculation, L Li, HH Xie, JS Zhao, \textcolor{cvblue}{Xiaoxiong Liu}, JB Deng, XR Hu, XM Tao, \textbf{Physical Review B} 96 (2), 024106, (2017)
  \item First-principle investigations of 3d transition metal (Fe, Cu, and Co)-doped rocksalt MgO by chain, \textcolor{cvblue}{Xiaoxiong Liu}, Q Gao, L Li, J Zhao, X Hu, J Deng, \textbf{Journal of Superconductivity and Novel Magnetism} 30 (6), 1635-1641, (2017)
  \item Effect of As and Nb doping on the magnetic properties for quaternary Heusler alloy FeCoZrGe, GY Mao, \textcolor{cvblue}{Xiaoxiong Liu}, Q Gao, L Li, HH Xie, G Lei, JB Deng, \textbf{Journal of Magnetism and Magnetic Materials} 398, 1-6, (2016)
  \item First-principle study of half-metallic ferromagnetism in rocksalt XO (X= Li, K, Rb, Cs), G Lei, \textcolor{cvblue}{Xiaoxiong Liu}, HH Xie, L Li, Q Gao, JB Deng,\textbf{Journal of Magnetism and Magnetic Materials} 397, 176-180, (2016)

\end{etaremune}









%----------------------------------------------------------------------------------------

\end{document}